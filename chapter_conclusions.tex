\chapter{Conclusions}
In this thesis have been described the efforts aimed at creating a
working implementation of the \cloudcast architecture, which have
culminated in the creation of the \cloudypeer framework. In achieving
this result a series of other accomplishments have been reached. First
and more importantly the original \cloudcast infrastructure has been
improved with the addition of a recovery mechanism directed to
guarantee the stability of the underlying \peersampling
protocol. Secondly a noteworthy endeavor has been targeted to resolve
bugs and extend the functionalities of the \grapes toolkit in order
to make it suitable to serve as foundation.

The general performances of the implementation have been analyzed and
confronted with the ones proposed in the original
paper~\cite{Cloudcast}. Moreover, the enhancements to the protocol's
algorithms have been evaluated, showing how the system benefits from
their presence and at the same time how the additional overhead
is overall negligible.

The \cloudypeer framework has been described in detail, explaining how
the various components interact with each other and how developers can
easily extend its behavior to fit their need. This analysis has also
served the purpose of proving the generality and ease of use features
central to the framework's design.

In conclusion it has been presented a concrete example application
called \cloudyrss, which proves the viability of
the proposed approach and the correctness of the described
implementation.

\paragraph{Future works}
There are a number of interesting problems that have not been covered
in this thesis due to lack of time. Concerning the \cloudcast
architecture itself, one of the results proposed in the original
paper~\cite{Cloudcast} has shown that a major portion of the expense is
caused by the \peersampling. An easy, yet effective, way to reduce its
impact would be to study alternative \textit{bootstrapping} strategies
not based on the \cloud. Examples of such tactics are:
\begin{itemize}
  \item Employing local and/or web based peer
    caches~\cite{GnutellaWebCache}~\cite{P2PVPN}
  \item Probing the network for entry
    points~\cite{BootstrappingP2P}~\cite{BootstrappingP2PLocality}
    ~\cite{DecentralizedBootstrappingP2P}
    \item Exploiting \textit{Dynamic DNS}
      services~\cite{DecentralizedBootstrapping}
\end{itemize}
Furthermore the system could be made more realistic by investigating
ways to take advantage of the security feature offered by \cloud
provider such as Amazon in terms of authentication, access control and
versioning~\cite{AmazonS3DevGuide}.

Moving our focus on the actual implementation, an obvious extension
would be the addition of more \cloud storage service
providers and a greater variety of \peersampling protocols. Moreover
further work should be made to better adapt the
\cloud \descriptors handling to the peculiarities of the \grapes's
cache design.

In conclusion, a final effort which would greatly enhance
the performances of the system is the development of a better \networkhelper
implementation for the \cloudypeer framework, possibly relaying on the
correspondent module of \grapes.
