\chapter{Introduction}
\textit{Cloud computing} represents an important tecnology in the
modern Internet era. The promises of virtually infinite storage
capacity and computation resources, in conjunction with the fact that
costs are computed with a \textit{pay-per-use} model, make it very
appealing to small companies. Recent years have seen the rise of Internet based
\textit{start-up}, which, by exploiting the \cloud, have the opportunity
to compete on equal terms with software giants like Microsoft
and Google. A representative example is Dropbox which, relaying on
\amazonsss service, has become, in a very short time, a dominant player
in the sector of file storage.

The increasing interest in \cloud computing has resulted in the
reduction of the appeal of another trend of the last decade: the \ptop
paradigm. This model displays similar characteristics with respect to
the \cloud, but, at the same time, it does presents substantial
differences. The
cost of operating a \ptop architecture is virtually non-existent,
making it extremely desirable from an economic point of view. At the
same time, however, the \cloud offers a much higher availability,
making it a more appropriate solution for those businesses which are not
well suited to the \textit{best effort} philosophy of \ptop.

Both the academic community and the commercial world have tried to mix
the two approaches with the intent of achieving the above mentioned
properties at the same time. The two main strategies which have
emerged see the
addition of peers to a centralized solution on one side and the
extension of existing \ptop systems with elastic computing nodes on
the other.
For example, the first tactic has been exploited in
video-on-demand~\cite{PeerAssistedVoD} and file
distribution~\cite{BitTorrentRobustness}~\cite{AmazingStore}
applications where storage
and bandwidth costs are reduced, when availability
constraints allow it, by off-loading to clients the burden of
providing the service. The second strategy, instead, has been applied
to bulk-synchronous content distribution~\cite{AngelsInCloud}
and online backup~\cite{PeerAssistedOnlineDataBackup} scenarios to
satisfy requirements beyond the reach of a \ptop system.

A novel approach to the problem, which greatly differentiates itself
from the previous examples, has been recently proposed. The
illustrated architecture, called \cloudcast~\cite{Cloudcast}, sees an
integration of the \cloud in all the level of the \ptop system, making
the two of them a coherent entity. To achieve this task, \cloudcast
exploits a number of standard \ptop protocols, some of which have been
modified to take into account the new requirements. The number of
entities involved in the architecture make developing applications
based on it a quite long and error-prone task. Other than focusing on
the target domain, a lot of work must go into designing and implementing
the base infrastructure.

This kind of problem is indeed true for many \ptop oriented solutions
and has been partially solved via the creation of multiple frameworks
by the academia~\cite{AntHill}~\cite{P2PFramework},
enterprise bodies~\cite{JXTA}~\cite{dotNETp2p} and open-source
projects~\cite{GNUnet}. The general idea is to simplify the burden
imposed on the developer, either by providing domain specific solutions
or by offering the building blocks needed to design a complex system.

\paragraph{Contributions} In this thesis we present the efforts that
have been made to develop a working implementation of the \cloudcast
infrastructure. Furthermore we describe and evaluate the improvements
that have been produced while performing the task. The second half of
the work deals with the analysis and the implementation of
\cloudypeer, a general \textit{Java} framework which, exploiting \cloudcast,
provides the means to easily build applications focused on the content
diffusion scenario. In conclusion, a practical example is presented in
the form of a \textit{RSS} delivery software, called \cloudyrss, which
has been developed to prove the viability of the proposed approach and
will be used, in future works, to evaluate it in a real context.
