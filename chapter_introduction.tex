\chapter{Introduction}

Developing application based on the \ptop paradigm is often a complex
and long task. Other than focusing on the particular application
domain problem a lot of work must go into designing and implementing
the base infrastructure upon which the target solution will function.

Most often the infrastructure is composed by standard protocols which
must be reinvented each and every time to fit the task at hand. Such
reinvention is in itself a bad habit but most importantly is a waste
of energy and talent of the application developer.

Multiple framework were developed both by the
academic~\cite{AntHill}~\cite{P2PFramework},
commercial~\cite{JXTA}~\cite{dotNETp2p} and open source~\cite{GNUnet}
world. Looking at the proposed example two main design philosophies
arise: on one side there are domain specific toolkit aimed at a
particular scenario while on the other end there are general framework
which only provide the very basic building block needed to build a
complex system.

\cloudypeer can be placed in the first category as it does provide
reusable components aimed at the development of information
dissemination applications. The novel characteristic of the framework
is to provide a seamless integration of a storage \cloud service to
the \ptop network.


\begin{itemize}
  \item Problem statement: economics of peer-to-peer with availability
    of the cloud
  \item Survey of existing solutions.
    Adding peer to centralized solution \cite{PeerAssistedVoD}
    \cite{BitTorrentRobustness}
    \cite{RapidCloudProvisioningLeaveragingP2P} \cite{AmazingStore}.

    Adding cloud to peer-to-peer system \cite{AngelsInCloud}, \cite{PeerAssistredOnlineDataBackup}

  \item Introduction of cloudcast
  \item Introduction of cloudypeer
  \item Introduction to the case study application cloudyrss
\end{itemize}

Note about the organization of following chapters.
